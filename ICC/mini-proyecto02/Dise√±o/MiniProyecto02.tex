\documentclass[a4paper,11pt]{report}
\usepackage[T1]{fontenc}
\usepackage[spanish]{babel}
\usepackage[utf8]{inputenc}
\usepackage{lmodern}

\title{\textbf{Mini-Proyecto 02}}
\author{Abrego Alvarez Jonathan \\ Martínez Anaya Luis Angel}

\begin{document}

\maketitle{\textbf{\large Mini-Proyecto 02: Diseño del programa}\\}

\textbf{Paso 1: Descubrir las clases necesarias para la solución del problema \\}
\begin{itemize}
\item Pokemon
\item Pokebola
\item Entrenador
\item Batalla
\item Ataque
\begin{itemize}
\item Ataque Normal
\item Ataque Continuo
\item Ataque Paralizador
\item Ataque Somnífero
\end{itemize}
\end{itemize}

\textbf{Paso 2: Determinar las responsabilidades de cada clase para la solución del problema}
\begin{itemize}
\item Pokemon: 

\begin{itemize}
\item Hay pokemones de varios tipos, cada uno tienen un tipo, se clasifican según su clase o especie, tienen un género, nivel y sus nombres de ataques son distintos
\item Un entrenador puede llamar a sus Pokemones como quiera, es decir, ponerles un apodo, de tal manera que puedan ser diferenciados en caso de que el entrenador tenga varios Pokemones de la misma especie.
\item Un Pokemon puede atacar a otro
\end{itemize}

\item Pokebola:

\begin{itemize}
\item Almacenar sólo un Pokemon dentro de ella
\item Permitir saber si está ocupada por otro Pokemon
\item Liberar al Pokemon que tiene contenido
\end{itemize}

\item Entrenador:
\begin{itemize}
\item Poseer seis pokemones contenidos en pokebolas en un cinturón
\item Tener un nombre, apellido, género, debe provenir de un lugar y pertenece a un equipo
\item Hacer que una pokebola libere al Pokemon que tiene contenido, así como hacer que el Pokemon vuelva a ocupar su pokebola
\item Conocer la información de los Pokemones que tiene en su cinturón
\item Participar en una batalla Pokemon
\end{itemize}

\item Batalla:
\begin{itemize}
\item Permitir que dos entrenadores liberen a un Pokemon
\item Pedir a un entrenador que ejecute una orden hacia su Pokemon liberado para atacar al del contrario
\item Hacer saber al entrenador que su Pokemon en batalla sigue sufriendo las consecuencias de un ataque especial hecho por el Pokemon del contrario
\item Determinar cuándo hay un ganador en la batalla, de lo contrario ésta debe continuar, cada entrenador con su turno
\end{itemize}

\item Ataque:
\begin{itemize}
\item Permitir que un pokemon pueda ejercer un ataque contra otro durante una batalla
\item Poder subir de nivel, con el fin de causar más daño al enemigo
\item Dependiendo del tipo de Pokemon sobre el cual se ejerce el ataque, si la clase del Pokemon es más vulnerable, el daño va a ser mayor que con cualquier otro Pokemon de distinta clase
\item Hacer una pequeña descripción del ataque que está realizando un Pokemon sobre otro
\item Dependiendo del nivel del Pokemon, la cantidad de daño que puede causar varía
\begin{itemize}
\item \textbf{Ataque Normal:}
\item Causar un daño estándar al Pokemon
\item \textbf{Ataque Continuo:}
\item Aparte del daño estándar al Pokemon, causar un daño extra por cada turno, cierto número de turnos
\item \textbf{Ataque Somnífero:}
\item Inhabilitar al Pokemon contrario de atacar, cierto número de turnos, con el fin de agarrar ventaja en la batalla
\item \textbf{Ataque Paralizador:}
\item Incapacitar al Pokemon liberado del entrenador contrario de poder realizar un ataque por cierto tiempo, mientras que el Pokemon contrario sí es capaz de seguirlo atacando.
\end{itemize}

\end{itemize}

\end{itemize}

\textbf{\\Paso 3: Determinar la colaboración entre las clases (Escenario)}
\begin{itemize}
\item Dar la bienvenida a dos usuarios que pretendan ser entrenadores Pokemon, y solicitarles datos
\item Pedirle a cada uno de los entrenadores que escojan los Pokemones que quieran tener en las seis pokebolas de su cinturón, escogerles un apodo a cada Pokemon que escojan
\item Una vez escogidos los Pokemones, desplegar un menú donde pueden saber información acerca de su registro como entrenadores, así como información de sus Pokemones, así como de proceder a la pelea
\item Pedir a cada uno de los entrenadores que liberen a uno de sus Pokemones para empezar la batalla
\item Pedirle a un entrenador que ordene a su Pokemon que realize cierto ataque sobre el Pokemon del entrenador contrario
\item Si en el ataque los puntos de vida del Pokemon contrario se acabaron, pedirle al segundo entrenador que libere a otro Pokemon de su cinturón para seguir con la batalla, si ya no tiene Pokemones disponibles, el segundo entrenador habrá perdido, de lo contrario, es turno de que ataque.
\end{itemize}
\end{document}
